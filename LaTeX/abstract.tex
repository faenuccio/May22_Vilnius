\documentclass[10pt,francais,a4paper]{amsart}
%\documentclass[10pt,english,a4paper]{amsart}
%%%%%%%%%%%%%%%%%%Typography packages
\usepackage[sc]{mathpazo}
\usepackage{eucal}
\usepackage{mathrsfs}
\usepackage[utf8]{inputenc}
\usepackage[T1]{fontenc}
\usepackage{hyperref}
\usepackage{indentfirst}
\usepackage[left=2.5cm,right=2.5cm, bottom = 3cm]{geometry}
\usepackage{xcolor}
%%%%%%%%%%%%%%%%%%Editing Packages
%\usepackage{cancel}
%\usepackage{soul}
%\usepackage{lineno}
%\linenumbers
%\usepackage[notref,notcite,color]{showkeys}
%%%%%%%%%%%%%%%%%%Babel
%\usepackage[french]{babel}
%\frenchsetup{StandardItemLabels=true}
\usepackage[english]{babel}
%%%%%%%%%%%%%%%%%%Math packages
\usepackage{amsmath,amsthm,amssymb}
\usepackage{mathtools}
\usepackage{enumitem}
\usepackage[all]{xy}
%%%Other commands
\newcommand{\Lean}{\texttt{LEAN-3}\kern.3em}
\newcommand{\mathlib}{\texttt{mathlib}\kern.3em}
%%%%
\title{A crash course about formalizing mathematics in \Lean}
\date{\today}
\author[Filippo A.~E.~Nuccio]{Filippo A. E. Nuccio Mortarino Majno di Capriglio}
\address{Univ Lyon, Université Jean Monnet Saint-Étienne \\ CNRS UMR 5208, Institut Camille Jordan \\ F-42023 Saint-Etienne \\ France}
\email{filippo.nuccio@univ.st-etienne.fr}
%\author{}
%\adress{}
%\email{}
\begin{document}
\maketitle
\thispagestyle{empty}
The aim of this series of lectures is to give a brief introduction to formalization of mathematics in \Lean, a \emph{proof assistant} which has recently proven to be particular well-suited for encoding modern mathematics.

A proof assistant is a piece of software that can interpret formal statements %in the form of definitions and theorems, 
and can assess the validity of the performed reasoning. Up until some years ago, the kind of mathematics that could possibly be taught to such a software was somewhat limited, and very far from modern research standards. This state of affair has been dramatically evolving in the last three-four years, and this reflects on a growing interest of the mathematical community
about these interactions with computer science. To mention a few remarkable examples, one can quote the work~\cite{BuzComMas20} where the authors formalized the definition of \emph{perfectoid space}, a remarkably complicated notion introduced in 2012 by the Fields medallist P.~Scholze; or the formalization challenge~\cite{Sch20} in 2020 (and completed in half a year) by P.~Scholze himself, about some recent result on \emph{condensed mathematics}; or the invitation of K.~Buzzard as plenary speaker at the ICM~2022 for a talk about formalized mathematics.

No previous knowledge of proof assistants is needed, nor assumed. For the three lectures on Tuesday, Wednesday and Thursday, it is required that you bring a laptop with an internet connection.
\section*{Contents}
\begin{description}
\item [May~$23^\text{rd}$, 3.00--5.00 PM, room 103] This will be a colloquium-style talk, aimed at introducing what proof assistants are, focusing on the assistant \Lean. I will present some easy example and relate about Scholze's challenge. After a break, in the final part I will informally introduce the \emph{Type Theory} underlying the mathematical foundations used in \Lean.
\item [May~$24^\text{th}$, 3.00--5.00 PM, room 312] In this first class, we will discuss logical connectors \mbox{$(\land,~\lor,~\iff)$} and basic tactics for interacting with \Lean. I will talk for 45--60 minutes, and then I will propose a series of exercises.
\item [May~$25^\text{th}$, 3.00--5.00 PM, room 312] In this second class, we will discuss basic properties of functions (surjectivity, injectivity, linear and affine functions) and how to express these properties in \Lean, toghether with some more advanced tactics. I will talk for 45--60 minutes, and then I will propose a series of exercises.
\item [May~$26^\text{th}$, 3.00--5.00 PM, room 312] In the third and final class, we will discuss how to introduce basic $\varepsilon/\delta$ reasoning in \Lean. I will talk for 45--60 minutes, and then I will propose a series of exercises.
\end{description}
%\section*{}.


\bibliography{abstract_biblio.bib}
\bibliographystyle{amsalpha}

\end{document}
